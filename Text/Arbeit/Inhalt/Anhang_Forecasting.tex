% TODO evtl umbennen 
\chapter{Forecasting - Technische Details}


% TODO
% Brier Score Erklärung
% Brier Score Probleme
%   Vorhersage unwahrscheinlicher Ereignisse nicht gewürdigt
%   50/50 fence sitters -> unbefriedigend, Frustration (need for closure, S. 81)
%   % S. 276 Indiscriminate fence-sitting
% Bayes Belief Updating Erklärung/Herleitung
% Kognitive Verzerrungen, Auflistung, Diskussion
%   Erlernbar oder nicht (- individuell nicht gelungen, + in Gruppe, 
%     Kommunikation hilft, Unterschiede nicht unüberwindbar)
% + eher statistisch, mathematische "Verzerrungen" (subadditivity ...)
% 11 Commandments GJP



\section{Ten Commandments of Superforecasting} \label{Ten}

Diese zehn Prinzipien für Forecasting können bei korrekter Anwendung die Genauigkeit von
Prognosen verbessern (vgl. \cite{Ten_Comm}). Eine kommentierte Version wird von Jackson und Reichin
angeboten (vgl. \cite{Jackson}, S.~292-294). Im folgenden Text werden diese zehn (eigentlich elf)
Prinzipien diskutiert:

\begin{description}

\item[(1) Triage:] \hfill \\
Diese Regel betrifft die Auswahl der Fragen, die man mit den Prognosen beantworten will und
der Ereignisse, die man vorhersagen will. So ist es nicht sinnvoll, sich mit sehr einfachen Fragen
oder Routinefragen zu beschäftigen, die mit Hilfe simpler Entscheidungsregeln beantwortet werden können.
Weiterhin macht es keinen Sinn, sich mit Fragen zu beschäftigen, die keine Relevanz haben. Schließlich
sollte man auch ein Gefühl dafür entwickeln, welche Prognosen notorisch fehleranfällig sind und damit
gemieden werden sollten.

Es ist auch wichtig auf die konkreten Fragestellungen bei den Prognosen zu achten, weil sich hier
eine Möglichkeit bietet, Vorhersagen zu verzerren. Die Art der Fragen und ihr \glqq{Schwierigkeitsgrad}\grqq{}
lassen sich nicht einschätzen, wenn lediglich der erreichte Brier Score zur Verfügung steht.
Unehrliche Forecaster können die erste Regel bewusst ausnutzen, indem sie die Auswahl der Fragen geschickt
manipulieren. Insbesondere können sie sich dadurch einen Vorteil gegenüber anderen Forecastern erschleichen,
die nicht den selben Satz an Fragen beantwortet haben. Es besteht grundsätzlich immer die Möglichkeit, dass
jemand den gleichen oder einen besseren Brier Score erreicht, weil er leichtere Fragen beantwortet hat.

\item[(2) Zerlegen von komplizierten Problemen in kleinere Teilprobleme:] \hfill \\
Diese Methode geht auf den Physiker Enrico Fermi zurück. Eine schwierige Frage, wie beispielsweise die
Frage nach der Anzahl der Klavierstimmer, die in Chicago leben, wird in kleinere Teilprobleme zerlegt:

\begin{description}

\item[(a)] Wie viele Klaviere gibt es in Chicago?
\item[(b)] Wie oft werden Klaviere jedes Jahr gestimmt?
\item[(c)] Wie lange dauert es ein Klavier zu stimmen?
\item[(d)] Wie viele Stunden im Jahr arbeitet ein Klavierstimmer durchschnittlich? 

\end{description}

Durch die Zerlegung in Teilprobleme werden getroffene Annahmen explizit gemacht, insbesondere können
fehlerhafte Annahmen besser erkannt werden. Schätzungen für die einzelnen Größen und damit für das Gesamtproblem
können erstellt und verschiedene Schätzungen miteinander verglichen werden. Dadurch kann der Grad der Unsicherheit
der Gesamtschätzung deutlicher herausgearbeitet werden.

\item[(3) Berücksichtigen von Basisraten:] \hfill \\
Dabei wird zwischen der Sicht von Außerhalb (\emph{outside view}) und der Sicht von Innerhalb (\emph{inside view})
unterschieden. Bei der Sicht von Außerhalb geht es darum, eine Basisrate für das spezifische Ereignis aufzustellen.
Die Kernfrage, die dabei beantwortet werden muss, lautet:

\begin{description}
\item[-] Wie oft geschehen Dinge dieser Art in ähnlichen Situationen?
\end{description}

Bei der Sicht von Innerhalb werden Eigenschaften gesammelt, die den konkreten Fall von vergleichbaren Fällen
unterscheiden. Diese neuen Variablen werden dann genutzt, um die Basisrate abzuändern und an die konkrete
Frage anzupassen.

Technischer ausgedrückt handelt es sich bei diesem Vorgehen um das Aufstellen eines Algorithmus, der auf Basisraten
basiert (siehe auch S.~\xcom). Zunächst wird anhand historischer Betrachtungen eine Basisrate aufgestellt, die
beschreiben soll, wie häufig Ereignisse dieser Art im Allgemeinen auftreten. Daraufhin werden Prädiktorvariablen gesucht, die
dafür verantwortlich sein können, dass die Häufigkeit eines Ereignisses von der allgemeinen Basisrate abweicht. Diese
Prädiktorvariablen werden dann gewichtet und mit der Basisrate verrechnet, wobei manche Prädiktoren das Ereignis wahrscheinlicher
machen, wohingegen andere die Warscheinlichkeit des betrachteten Ereignisses geringer machen.

\item[(4) Praktizieren von Belief Updating:] \hfill \\
Gute Forecaster passen ihre Weltsicht ständig an neue Erkenntnisse an. Dabei geht es nicht nur darum die Formel von Bayes
(siehe S.~\xcom) anzuwenden. Vielmehr müssen zunächst wichtige Signale aus dem Fluss von Nachrichten herausgefiltert werden, wobei
Wunschdenken, Über- und Unterreaktionen vermieden werden müssen. Hierfür müssen Forecaster ständig die aktuellen Nachrichten beobachten 
und kritisch widerspiegeln.

\item[(5) Berücksichtigen rivalisierender Weltsichten:] \hfill \\
Hierbei geht es darum, den Wahrheitsgehalt zweier gegensätzlicher Weltanschauungen (Hypothesen) in Verhältnis zueinander zu bringen (siehe S.~\xcom).
Je höher der Wahrheitsgehalt einer Hypothese, desto eher ist sie für die Erstellung von Prognosen relevant. Insbesondere
sollte allein die eigene Weltsicht zur Bildung von Vorhersagen herangezogen werden, wenn man sehr sicher ist, dass die Weltsicht des
geistigen Rivalen absolut falsch ist. Zudem kann das Verhältnis der Hypothesen sich angesichts neuer Ereignisse verändern.
Treten Ereignisse ein, die mit Hilfe einer Weltanschauung mit höherer Wahrscheinlichkeit vorhergesagt wurden, dann erhält diese
Hypothese zukünftig mehr Gewicht.

\item[(6) Berücksichtigen von Unsicherheit:] \hfill \\
Forecaster müssen lernen, den Grad der Unsicherheit, den sie bei der Fällung ihrer Urteile verspüren, zu quantifizieren.
Vage Ausdrücke wie \glqq{wahrscheinlich}\grqq{}, \glqq{vielleicht}\grqq{} oder \glqq{kaum vorstellbar}\grqq{} müssen in
Zahlen, also subjektiven Wahrscheinlichkeiten, ausgedrückt werden.

\item[(7) Finden einer Balance zwischen Unentschlossenheit und Verbissenheit:] \hfill \\
Zum Einen geht es bei diesem Ratschlag um das Dilemma, das entsteht, wenn man versucht seine Vorhersagen im Hinblick auf
den Kalibrierungs- und Diskriminierungswert des Brier Score (siehe Anhang~\xcom) zu verbessern. Zu viel Entschlossenheit
kann zu großen Fehlern und folglich zu Einbußen bei der Kalibrierung führen. Unentschlossenheit hingegen kann dem
Diskriminierungswert schaden, da bei vielen Prognosen eine vorsichtige \glqq{50/50-Schätzung}\grqq{} gegeben wird. Um
die Vorhersagegenaugkeit auf einem guten Niveau zu halten, muss hierbei eine Balance gefunden werden.

Zum Anderen spielt es auch eine Rolle, wie ein Urteil oder eine Prognose kommuniziert wird. Ein Urteil, das starke
Unsicherheit ausdrückt, kann entmutigend wirken. Aus diesem Grund wird vorgeschlagen, nach der Urteilsfindung zwar
eine entschlossene Haltung einzunehmen aber trotzdem den Grad der Unsicherheit bei der Entscheidung zu vermitteln
(vgl. \cite{Jackson}, S.~293).

\item[(8) Aus Fehlern lernen:] \hfill \\
Ein Vorschlag, der als sehr schwierig in der Umsetzung gilt (vgl. \cite{Jackson}, S.~294). Gute Forecaster müssen
ehrliche Lektionen aus ihren Fehlern ziehen, anstatt durch Hindsight Bias (siehe S.~\xcom) und Rechtfertigungsversuche
(Belief System Defenses, siehe S.~\xcom) davon abgelenkt zu werden.

\item[(9) Gruppenarbeit lernen:] \hfill \\
In Laufe des Good Judgement Projects wurde beobachtet, dass Gruppen bessere Prognose erstellen als einzelne Personen
(vgl. \cite{Jackson}, S.~294). Allerdings müssen für eine solche konstruktive Gruppenarbeit einige Fähigkeiten einstudiert
werden, insbesondere:

\begin{description}

\item[(a) Perspektivenwechsel (\emph{perspective taking}):] \hfill \\
Die Fähigkeit andere Perspektiven einzunehmen und die Argumente der Gegenseite zu verstehen.

\item[(b) Präzise Fragestellungen (\emph{precision questioning}):] \hfill \\
Anderen bei der Erläuterung ihrer Argumente helfen, um Missverständnisse zu vermeiden.

\item[(c) Konstruktive Konfrontation (\emph{constructive confrontation}):] \hfill \\
Die Fähigkeit, jemandem auf freundliche Weise zu widersprechen.

\end{description}

\item[(10) Üben:] \hfill \\
Beim Forecasting läuft man oft Gefahr beim Korrigieren eines Fehlers einen anderen zu begehen. Aus diesem Grund brauchen
Forecaster viel Übung und ausreichendes Feedback, um die notwendige Balance zu finden\footnote{
Good Judgement Inc. betreibt eine Online-Plattform, die Freiwilligen die Möglichkeit bietet, Prognosen abzugeben und ihre
Forecasting Fähigkeiten zu verbessern (\url{https://www.gjopen.com/}).
}.

\item[(11) Kritisch bleiben:] \hfill \\
Zum Schluss wird betont, das die Regeln eher als Ratschläge interpretiert werden sollten, die auch jederzeit hinterfragt
werden können. Sonst läuft man Gefahr, in zu starre Denkmuster zu verfallen.
%---------------------------------------------------------------------------------
\end{description}

Wie in der Beschreibung angedeutet, versuchen manche Ratschläge, mathematische Formulierungen und deren Auswirkungen in Worten zu beschreiben.
Es ist jedoch einfacher, die Ratschläge zu verstehen und vor Allem auch einzuhalten, wenn die zugrunde liegenden Formeln bekannt
sind.
\newpage

\section{Empirische Korrespondenz - Der Brier Score} \label{Anhang_Brier}

\subsection{Brier Score}

Gleichung \ref{brier} definiert den Brier Score (\cite{Tetlock}, S. 273).

\begin{equation}
\textrm{PS} = \frac{\sum_{i=1}^{M} (p_i - x_i)^2}{ M }
\label{brier}
\end{equation}

\begin{description}
\item[$M$ :] Anzahl der gemachten Vorhersagen
\item[$p_i$ :] subjektive Wahrscheinlichkeit für den Eintritt von Ereignis $i$
\item[$x_i$ :] die tatsächliche Realisierung von Ereignis $i$ (0 wenn das Ereignis ausgeblieben ist, 1 wenn das Ereignis eingetreten ist)
\end{description}

\subsection{Brier Score - Zerlegung}

Weiterhin zeigt Gleichung \ref{brier_decomposition} die Zerlegung des Brier Score (\cite{Tetlock}, S. 274).

\begin{equation}
\textrm{PS} = \underbrace{
b(1-b)
}_\textrm{VI} + \underbrace{
\frac{1}{N} \sum_{t=1}^{T} n_t (p_t - b_t)^2
}_\textrm{CI} - \underbrace{
\frac{1}{N} \sum_{t=1}^{T} n_t (b_t - b)^2
}_\textrm{DI}
\label{brier_decomposition}
\end{equation}

\begin{description}
\item[$b$ :] Basisrate aller Ereignisse
\item[$b_t$ :] Basisrate für die Vorhersagekategorie $t$
\item[$N$ :] Gesamtanzahl der Ereignisse
\item[$n_t$ :] Anzahl der Vorhersagen in Kategorie $t$
\item[$T$ :] Anzahl der Vorhersagekategorien
\item[$p_t$ :] Vorhersage von Kategorie $t$
\end{description}

Weiterhin ist VI die Variability, CI die Calibration, DI die Discrimination.

\subsection{Rechenbeispiel}


\section{Logische Korrespondenz - Bayes Belief Updating} \label{Anhang_Bayes}

