% TODO evtl umbennen 
\chapter{Forecasting - Technische Details}

Für Überprüfbarkeit:

Vollständig: Es gibt keine weitere Kategorie, in die ein Objekt (oder in diesem
Fall ein Ereignis) fallen könnte.

Gegenseitig ausschließend: ...

Die menschliche Intuition ist schlecht in Wahrscheinlichkeitsrechnung.

% TODO
% Brier Score Erklärung
% Brier Score Probleme
%   Vorhersage unwahrscheinlicher Ereignisse nicht gewürdigt
%   50/50 fence sitters -> unbefriedigend, Frustration (need for closure, S. 81)
%   % S. 276 Indiscriminate fence-sitting
% Bayes Belief Updating Erklärung/Herleitung
% Kognitive Verzerrungen, Auflistung, Diskussion
%   Erlernbar oder nicht (- individuell nicht gelungen, + in Gruppe, 
%     Kommunikation hilft, Unterschiede nicht unüberwindbar)
% + eher statistisch, mathematische "Verzerrungen" (subadditivity ...)
% 11 Commandments GJP

\section{Ten Commandments of Superforecasting}

Diese zehn Prinzipien für Forecasting können bei korrekter Anwendung die Genauigkeit von
Prognosen verbessern (vgl. \cite{Ten_Comm}). Eine kommentierte Version wird von Jackson und Reichin
angeboten (vgl. \cite{Jackson}, S.~292-294). Im folgenden Text werden diese zehn (eigentlich elf)
Prinzipien diskutiert:

\begin{description}

\item[(1) Triage:] \hfill \\
Diese Regel betrifft die Auswahl der Fragen, die man mit den Prognosen beantworten will und
der Ereignisse, die man vorhersagen will. So ist es nicht sinnvoll, sich mit sehr einfachen Fragen
oder Routinefragen zu beschäftigen, die mit Hilfe simpler Entscheidungsregeln beantwortet werden können.
Weiterhin macht es keinen Sinn, sich mit Fragen zu beschäftigen, die keine Relevanz haben. Schließlich
sollte man auch ein Gefühl dafür entwickeln, welche Prognosen notorisch fehleranfällig sind und damit
gemieden werden sollten.

Es ist auch wichtig auf die konkreten Fragestellungen bei den Prognosen zu achten, weil sich hier
eine Möglichkeit bietet, Vorhersagen zu verzerren. Die Art der Fragen und ihr \glqq{Schwierigkeitsgrad}\grqq{}
lassen sich nicht einschätzen, wenn lediglich der erreichte Brier Score zur Verfügung steht.
Unehrliche Forecaster können die erste Regel bewusst ausnutzen, indem sie die Auswahl der Fragen geschickt
manipulieren. Insbesondere können sie sich dadurch einen Vorteil gegenüber anderen Forecastern erschleichen,
die nicht den selben Satz an Fragen beantwortet haben. Es besteht grundsätzlich immer die Möglichkeit, dass
jemand den gleichen oder einen besseren Brier Score erreicht, weil er leichtere Fragen beantwortet hat.

\item[(2) Zerlegen von komplizierten Problemen in kleinere Teilprobleme:] \hfill \\
Diese Methode geht auf den Physiker Enrico Fermi zurück. Eine schwierige Frage, wie beispielsweise die
Frage nach der Anzahl der Klavierstimmer, die in Chicago leben, wird in kleinere Teilprobleme zerlegt:

\begin{description}

\item[(a)] Wie viele Klaviere gibt es in Chicago?
\item[(b)] Wie oft werden Klaviere jedes Jahr gestimmt?
\item[(c)] Wie lange dauert es ein Klavier zu stimmen?
\item[(d)] Wie viele Stunden im Jahr arbeitet ein Klavierstimmer durchschnittlich? 

\end{description}

\end{description}