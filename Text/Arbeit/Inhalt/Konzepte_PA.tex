\chapter{Konzepte von Predictive Analytics}
\label{part:Konzepte_PA}

\section{Begriffsdefinition und -Abgrenzung}

\section{Probleme menschlicher Urteile}

\section{Bestandteile von \emph{predictive analytics}}

Bei \emph{predictive analytics} werden Daten mit Hilfe von geeigneten Methoden
verarbeitet, um verschiedene Arten von Prognosen zu ermöglichen. Dem
Datenanalysten stehen für diese Aufgabe Computeranwendungen zur Verfügung.
Diese Programme lesen die Daten ein, führen die notwendigen Berechnungen aus und
ermöglichen eine graphische Aufbereitung der Ergebnisse. \\ \\
Standardisierte
Vorgehensmodelle können helfen, die Arbeitsschritte einer Datenanalyse zu
vereinheitlichen. Ein Beispiel für ein solches Vorgehensmodell ist CRISP-DM \xcom.

\subsection{CRISP-DM Vorgehensmodell}

\subsection{Daten}

\subsubsection{Elementare Datentypen}

Bei Daten lassen sich verschiedene elementare Typen\footnote{Alternativ dazu
spricht man vom Skalenniveau (\emph{scale of measurement}) als eine
Eigenschaft von Daten und Variablen.} unterscheiden.
Je nach Datentyp einer
Variablen, werden ihre Werte unterschiedlich interpretiert. Zudem können
bestimmte mathematische Operationen nur mit Variablen eines bestimmten Typs
durchgeführt werden. Die Datentypen werden im folgenden Text genauer erläutert
(vgl. \cite{Arens}, S.~1229 und \cite{McCarthy}, S.~28-29).\\ \\
Der \glqq{einfachste}\grqq Datentyp ist eine nominal skalierte Variable. 
Werte von nominal skalierten Variablen werden als Namen
interpretiert. Weiterhin ist es nicht sinnvoll diese Art von Variablen zu
ordnen. Es lässt sich lediglich feststellen, ob zwei Variablen
gleich oder ungleich sind. Ein Beispiel für eine nominal skalierte Variable ist
der Name einer Stadt. Die Werte sind dann konkrete Städtenamen wie
\glqq{München}\grqq oder \glqq{Berlin}\grqq. Als Operationen stehen nur $=$
oder $\neq$ zur Verfügung, weil Aussagen wie 
\glqq{München}\grqq$=$\glqq{München}\grqq oder
\glqq{München}\grqq$\neq$\glqq{Berlin}\grqq sinnvoll sind. Nicht sinnvoll wären
dagegen Vergleiche wie \glqq{München}\grqq$<$\glqq{Berlin}\grqq
\footnote{Der Vergleich wäre sinnvoll, wenn mit der Nennung des Namens 
z. B. implizit die Größe der Stadt gemeint wäre. Dies ist hier aber nicht der
Fall.}.
Weitere Beispiele für nominal skalierte Variablen sind Geschlecht, Augenfarbe
oder Postleitzahl. \\ \\
Etwas mehr Möglichkeiten stehen zur Verfügung, wenn eine ordinal skalierte
Variable vorliegt. Für eine solche Variable ist eine Ordnung sinnvoll, sodass
alle Vergleichsoperationen möglich sind. Eine Kleidungsgröße (\texttt{S},
\texttt{M}, \texttt{L}) ist ein Beispiel für eine ordinal skalierte
Variable. Im Gegensatz zum Beispiel mit den Städtenamen ist ein Vergleich wie
\texttt{S} $<$ \texttt{M} hier sinnvoll. \\ \\
Wird eine Variable auf einer Intervallskala definiert, sind ihre Werte
numerisch.
Zusätzlich zu
allen Operationen von nominal und ordinal skalierten Variablen können hier auch
Additionen und Subtraktionen ausgeführt werden. Ein Beispiel hierfür sind
Temperaturen in °C, die addiert und subtrahiert werden können. Allerdings sind
20~°C nicht das Doppelte von 10~°C. Hier kann man erkennen, dass es bei
Intervallskalen nicht sinnvoll ist, Verhältnisse zu berechnen. Der Grund
hierfür ist, dass der Nullpunkt einer Intervallskala nicht mit dem absoluten
Nullpunkt einer Größe identisch sein muss. So sind 0~°C nicht der absolute
Nullpunkt für die Temperatur. Wenn man sinnvolle Verhältnisse berechnen will,
wird eine Proportionalskala benötigt. \\ \\
Eine auf einer Proportionalskala definierte Variable hat numerische Werte, für
die alle zuvor erwähnten mathematischen Operationen definiert sind und
zusätzlich auch Multiplikation und Division möglich sind.
%Insbesondere ist es sinnvoll Verhältnisse zu berechnen.
So führt eine
Multiplikation einer Länge in Metern mit der Konstante 2 zu der doppelten Länge.
Ist das Verhältnis zweier Längen gleich 10, so ist die eine Länge 10
mal so groß wie die andere. Weiterhin führt die Multiplikation zweier Längen
in m zu einer korrekten Fläche in $\textrm{m}^2$. Die Ergebnisse wären nicht
korrekt, wenn der Nullpunkt der Meterskala nicht mit dem Nullpunkt der
Längenskala identisch wäre, wie es bei Intervallskalen der Fall ist. \\ \\
Nominal und ordinal skalierte Variablen sind qualitative Variablen.
Variablen, die auf Intervall- oder Proportionalskalen definiert sind, werden
hingegen als quantitative oder kardinale Variablen bezeichnet.
Weiterhin werden qualitative Variablen auch als kategorisch
(\emph{categorical}) bezeichnet, quantitative als numerisch (\emph{numeric})
und je nach Wertebereich als diskret (\emph{discrete}, ganzzahlige Werte) oder
kontinuierlich (\emph{continuous}, Fließkommawerte). \\ \\
Tabelle~\ref{tab:Skalen} zeigt eine Zusammenfassung der verschiedenen Datentypen
und der erlaubten Operationen\footnote{
Basierend auf Tabelle 2.2 in \cite{Runkler}, S.~8}.
\begin{table}
%\footnotesize
\centering
\caption{Übersicht der Datenskalen}
\label{tab:Skalen}
\scalebox{0.7}{
\begin{tabular}{ |l|l|l|l|  }
\hline
Skala & Sinnvolle Operationen & Beispielgrößen & Beispielwerte \\
\hline
Nominal & Gleichheit ($=, \neq$) & Name, & \glqq{Julia}\grqq,
  \glqq{Klaus}\grqq \\
& & Geschlecht & \texttt{M}, \texttt{W} \\
\hline
Ordinal & Gleichheit ($=, \neq$), & Kleidungsgöße & \texttt{S}, \texttt{M},
  \texttt{L} \\
        & Vergleiche ($<, >, \ldots$) & & \\
\hline
Intervall & Gleichheit ($=, \neq$), & Jahresangabe, & 2015~A.D. \\
  & Vergleiche ($<, >, \ldots$), & Temperatur in Grad Celsius & 20~°C \\
  & Addition ($+$), Subtraktion ($-$), &  &  \\
\hline
Proportional & Gleichheit ($=, \neq$), & Alter, & 21 Jahre \\
  & Vergleiche ($<, >, \ldots$), & Temperatur in Kelvin & 273.4~K \\
  & Addition ($+$), Subtraktion ($-$), &  &  \\
  & Multiplikation ($\cdot$), Division ($/$) & & \\
\hline

%\multicolumn{3}{|c|}{Country List} \\
%\hline
%Country Name     or Area Name& ISO ALPHA 2 Code &ISO ALPHA 3 \\
%\hline \hline
%Afghanistan & AF &AFG \\
%Aland Islands & AX   & ALA \\
%Albania &AL & ALB \\
%Algeria    &DZ & DZA \\
%American Samoa & AS & ASM \\
%Andorra & AD & AND   \\
%Angola & AO & AGO \\
%\hline
\end{tabular}
}
\end{table}


\subsubsection{Information Management (Data Warehouse)}

\subsubsection{Datenabhängige Ziele}

\paragraph{\ldots}

\paragraph{Beschreibung der Daten}

\paragraph{Klassifikation}

\paragraph{Zeitreihenanalyse}

\subsection{Methoden}

\subsubsection{Die Großen Drei}

\paragraph{Regression}

\paragraph{Entscheidungsbäume}

\paragraph{Neuronale Netze}

\subsubsection{Zeitreihenanalyse}

\subsubsection{\ldots}

\subsection{Werkzeuge}

\subsection{(Kreative) Freiheitsgrade bei der Implementierung eines
  Vorhersagemodells}

\section{Anwendungsbeispiele}

\subsection{Allgemeine Anwendungsfelder (siehe SAP Artikel)}

\subsection{\ldots}

\section{Allgemeine Risiken bei der Nutzung von Predictive Analytics}

Es exisitieren Risiken, die dazu führen können, dass die Ziele einer
Datenanalyse nicht oder nicht in vollem Umfang erreicht werden können.
Im folgenden Text werden einige dieser Risiken erläutert.

\subsection{Risiko der Unverhältnismäßigkeit}

Die Anwendung von Predictive Analytics ist mit einem hohen Aufwand verbunden.
Dabei spielen die Kosten für die Datenerhebung eine wesentliche Rolle.
Zusätzlich werden für die Datenanalyse Kenntnisse aus verschiedenen
Fachrichtungen benötigt. So sind einerseits anwendungsspezifische Kenntnisse
zur Interpretation der Daten und der Ergebnisse wünschenswert. Andererseits
werden zur Durchführung der Datenanalyse Kenntnisse in Mathematik, Statistik und
Informatik benötigt. \\ \\
Aus diesem Grund besteht das Risiko, dass die Kosten einer Anwendung von
Predictive Analytics den Nutzen übersteigen. Es ist auch möglich, dass das
gleiche Ergebnis mit einer einfacheren, kostengünstigeren Methode erreicht
werden kann. In diesem Fall würde die Anwendung eines aufwändigen Predictive
Analytics Verfahrens wertvolle Ressourcen binden, die an anderer Stelle stärker
gebraucht werden. \\ \\
Somit ist es wichtig, Betrachtungen zu Alternativkosten \todo{gls eintrag}
in die Planung von \emph{predictive analytics} Anwendungen einzubeziehen. 

\subsection{Risiken bei der Datenerhebung}

Trainingsdaten spiegeln nicht (mehr) das Verhalten des Systems wider.

\subsubsection{\ldots}

\subsection{Risiken bei der Interpretation der Daten}

\subsection{Risiken bei der Interpretation der Ergebnisse}

Wenn die Ergebnisse der Datenanalyse zur Entscheidungsunterstützung herangezogen
werden, beeinflussen sie das Verhalten der Entscheidungsträger. Dies kann zur
Entstehung problematischer psychosozialer Effekte führen. Zwei Beispiele hierfür
werden nun erläutert.

\subsubsection{Prognose verändert das Verhalten des Systems }

Die Ergebnisse von Vorhersagen werden von Menschen interpretiert,
die daraufhin ihr Verhalten anpassen. Dies kann zu Rückkopplungseffekten führen,
die von negativen Auswirkungen im Sinne einer
\glqq{Selffulfilling} Prophecy\grqq begleitet werden können
(vgl. \cite{Crossman}). Bestimmte Prognosen können beispielsweise von einer 
Interessengruppe als Bestätigung ihrer Agenda interpretiert werden, wobei
anderslautende Vorhersagen ignoriert werden. Dadurch bestärkt, setzt die Gruppe
ihre gewünschten Handlungsoptionen um. Dies ruft den vorhergesagten Effekt
jedoch erst hervor.

\subsubsection{Ignorieren der Unsicherheiten bei der Prognose}

Es besteht die Gefahr, dass die Unsicherheiten von Vorhersagen ignoriert werden
und die Prognose als eine Gewissheit betrachtet wird. Somit wird möglichen,
alternativen Entwicklungen bei der Entscheidungsfindung nicht genügend Bedeutung
beigemessen. Dies kann dazu führen, dass Risiken falsch kalkuliert werden und
in der Zukunft nicht genügend Handlungsoptionen zur Verfügung stehen.
