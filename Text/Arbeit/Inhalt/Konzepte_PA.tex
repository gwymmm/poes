\chapter{Konzepte von Predictive Analytics}
\label{part:Konzepte_PA}

\section{Begriffsdefinition und -Abgrenzung}

\section{Bestandteile von Predictive Analytics}

\subsection{CRISP-DM Vorgehensmodell}

\subsection{Daten}

\subsubsection{Information Management (Data Warehouse)}

\subsubsection{Datenabhängige Ziele}

\paragraph{\ldots}

\paragraph{Beschreibung der Daten}

\paragraph{Klassifikation}

\paragraph{Zeitreihenanalyse}

\subsection{Methoden}

\subsubsection{Die Großen Drei}

\paragraph{Regression}

\paragraph{Entscheidungsbäume}

\paragraph{Neuronale Netze}

\subsubsection{Zeitreihenanalyse}

\subsubsection{\ldots}

\subsection{Werkzeuge}

\subsection{(Kreative) Freiheitsgrade bei der Implementierung eines
  Vorhersagemodells}

\section{Anwendungsbeispiele}

\subsection{Allgemeine Anwendungsfelder (siehe SAP Artikel)}

\subsection{\ldots}

\section{Allgemeine Risiken bei der Nutzung von Predictive Analytics}

Es exisitieren Risiken, die dazu führen können, dass die Ziele einer
Datenanalyse nicht oder nicht in vollem Umfang erreicht werden können.
Im folgenden Text werden einige dieser Risiken erläutert.

\subsection{Risiko der Unverhältnismäßigkeit}

Die Anwendung von Predictive Analytics ist mit einem hohen Aufwand verbunden.
Dabei spielen die Kosten für die Datenerhebung eine wesentliche Rolle.
Zusätzlich werden für die Datenanalyse Kenntnisse aus verschiedenen
Fachrichtungen benötigt. So sind einerseits anwendungsspezifische Kenntnisse
zur Interpretation der Daten und der Ergebnisse wünschenswert. Andererseits
werden zur Durchführung der Datenanalyse Kenntnisse in Mathematik, Statistik und
Informatik benötigt. \\ \\
Aus diesem Grund besteht das Risiko, dass die Kosten einer Anwendung von
Predictive Analytics den Nutzen übersteigen. Es ist auch möglich, dass das
gleiche Ergebnis mit einer einfacheren, kostengünstigeren Methode erreicht
werden kann. In diesem Fall würde die Anwendung eines aufwändigen Predictive
Analytics Verfahrens wertvolle Ressourcen binden, die an anderer Stelle stärker
gebraucht werden. \\ \\
Somit ist es wichtig, Betrachtungen zu Alternativkosten \todo{gls eintrag}
in die 
Planung von Predictive Analytics Anwendungen einzubeziehen. 

\subsection{Risiken bei der Datenerhebung}

\subsubsection{\ldots}

\subsection{Risiken bei der Interpretation der Daten}

\subsection{Risiken bei der Interpretation der Ergebnisse}

Wenn die Ergebnisse der Datenanalyse zur Entscheidungsunterstützung herangezogen
werden, beeinflussen sie das Verhalten der Entscheidungsträger. Dies kann zur
Entstehung problematischer psychosozialer Effekte führen. Zwei Beispiele hierfür
werden nun erläutert.

\subsubsection{Prognose verändert das Verhalten des Systems 
%  (``Selffulfilling Prophecy'')}
  (\glqq{Selffulfilling} Prophecy\grqq)}

\subsubsection{Ignorieren der Unsicherheiten bei der Prognose}

Es besteht die Gefahr, dass die Unsicherheiten von Vorhersagen ignoriert werden
und die Prognose als eine Gewissheit betrachtet wird. Somit wird möglichen,
alternativen Entwicklungen bei der Entscheidungsfindung nicht genügend Bedeutung
beigemessen. Dies kann dazu führen, dass Risiken falsch kalkuliert werden und
in der Zukunft nicht genügend Handlungsoptionen zur Verfügung stehen.
